\documentclass[letterpaper,10pt]{article}
\usepackage[top=2cm, bottom=1.5cm, left=1cm, right=1cm]{geometry}
\usepackage{amsmath, amssymb, amsthm,graphicx}
\usepackage{fancyhdr}
\pagestyle{fancy}

\lhead{\today}
\chead{Quality Engineering Assignment 2}
\rhead{Justin Hood}

\newcommand{\Z}{\mathbb{Z}}
\newcommand{\Q}{\mathbb{Q}}
\newcommand{\R}{\mathbb{R}}
\newcommand{\C}{\mathbb{C}}
\newtheorem{lem}{Lemma}

\begin{document}
\begin{description}
\item[Problem 1.1]\hfill \\
It is difficult to define quality for two main reasons. The first is that the meaning of quality, and what consumers have considered to be the hallmarks of a quality product have changed and evolved over time. The second is that the descriptors of quality are often quite challenging to quantify and measure in a precise sense. This is why we use the so called ``dimensions of quality" to more accurately define it.
\item[Problem 1.6]\hfill \\
The internal customers of a business are those that interact with said business, while not necessarily being a direct recipient of the goods or services of the company. These customers can be employees that have some measure of profit sharing or stock incentive, as well as share holders. These customers are very important to a business, as they often have valuable insights and a direct link to the quality of the products themselves.
\item[Problem 1.9]\hfill \\
The three technical tools for quality control and improvement are,
\begin{enumerate}
\item Design of Experiments
\item Acceptance Sampling
\item Statistical Process Control
\end{enumerate}
\item[Problem 1.14]\hfill \\
Internal failure costs tend to be more important than external failure costs, as the removal of internal failures leads to a reduction in external failures. Because these internal failure costs tend to be less than the cost of a customer receiving a bad product, these are more important to reduce.
\item[Problem 1.15]\hfill \\
A six sigma process is a process who's variability is small enough that its specification limits are at least six standard deviations from the nominal value.
\item[Problem 1.19]\hfill \\
I agree with this statement. Without top management leadership, large scale processes have a great risk of failure. In addition to this, proper leadership incentivizes good employees to do even better and take quality improvement even higher, while improper management hurts quality improvement.
\item[Problem 2.2]\hfill \\
Risk plays an important role in the selection and define step of DMAIC. Because projects using DMAIC have preordained start and end dates, as well as a defined scope, project leaders need to account for variation and unknown setbacks in the projects length and scope. As the project continues, the risk associated should fall, as more information and better planning becomes available.
\item[Problem 2.3]\hfill \\
The net financial gain from this project is the overall income (or savings) less the cost to complete the project, i.e.
\[\$Ax-\$C=N\]
If $N$ is a positive value, we see that economically the project will have a positive impact on the business.
\item[Problem 2.9]\hfill \\
In the operation of a hospital emergency room, I would identify potential KPIV's to be,
\begin{itemize}
\item Functional Equipment and Supplies
\item Adequate staff size
\item Proper training for staff
\item Proper budget for maintianing a positive work culture of success
\end{itemize}
I would identify the KPOV's to be,
\begin{itemize}
\item Accurate Diagnosis/Treatment
\item Speed of Diagnosis/Treatment
\item High levels of patient satisfaction with treatment
\end{itemize}
Of these, I feel like they are all CTQ to the patients in the ER. The only tenative CTQ could be the budget, but this is just an important KPIV that is removed from the patients directly.
\item[Problem 2.12]\hfill \\
Our equation is,
\[3.4=6210(1-.50)^x\]
Solving, we find $x$ to be,
\[x=10.83484\text{years}\]
\end{description}
\end{document}

\documentclass[letterpaper,10pt]{article}
\usepackage[top=2cm, bottom=1.5cm, left=1cm, right=1cm]{geometry}
\usepackage{amsmath, amssymb, amsthm,graphicx}
\usepackage{fancyhdr}
\pagestyle{fancy}

\lhead{\today}
\chead{INMGT 622 Week 1 Assignment}
\rhead{Justin Hood}

\newcommand{\Z}{\mathbb{Z}}
\newcommand{\Q}{\mathbb{Q}}
\newcommand{\R}{\mathbb{R}}
\newcommand{\C}{\mathbb{C}}
\newtheorem{lem}{Lemma}

\begin{document}
\begin{description}
\item[Problem 1] \hfill \\
\begin{enumerate}
\item A shipment from Vendor 1 consists of 10000 parts, wherein we assume that 1.5\% are defective. Thus we conclude that on an average shipment from Vendor 1 the shipment will have 9850 working parts and 150 defective parts. We then consider the conditions for a sample to pass inspection. As stated in the prompt, the only success conditions are when
\[\#\ of\ defectives < 2\]
Thus, we consider how to compute the probability of this. To compute whether a certain number of defective parts is in a random sample of 100, we construct a hypergeometric distribution. By the independence of the distribution, we may write,
\[P(X<2)=\sum_{i=0}^1P(X=i)\]
The PMF of the hypergeometric distribution is,
\[P(X=x)=\frac{\binom{R}{x}\binom{N-R}{n-x}}{\binom{N}{n}}\]
For our problem,
\[P(X=x)=\frac{\binom{150}{x}\binom{9850}{100-x}}{\binom{10000}{100}}\]
Thus,
\[P(pass)=\sum_{i=0}^1P(X=i)=\frac{\binom{150}{0}\binom{9850}{100}}{\binom{10000}{100}}+\frac{\binom{150}{1}\binom{9850}{99}}{\binom{10000}{100}}\]
Using $R$, we find this to be,
\[P(pass)=0.218941+0.336798=0.555739\]
Thus, we see that the probability of accepting a shipment from Vendor 1 is approximately $55.6\%$.
\item On the other hand, a shipment from Vendor 2 consists of 5000 parts, wherein we assume that 0.5\% are defective. Thus, an average shipment from Vendor 2 will have 4975 working parts, and 25 defective parts. We then consider the conditions for a shipment to fail inspection. As stated a fail condition occurs when,
\[\#\ of\ defectives \geq 2\]
Under this sampling plan, we will again use the hypergeometric distribution and its independence to write,
\begin{align*}
1 &= \sum_i P(X=i),\ i\in 0,1,2,\ldots,100\\
 & \Leftrightarrow\\
1-(P(X=0)+P(X=1)) &= \sum_{i=2}^{100}P(X=i)
\end{align*}
I.e., $1-P(pass)=P(fail)$. Using the PMF for the hypergeometric distribution, we write, 
\[P(reject)=1-\frac{\binom{25}{0}\binom{4975}{100}}{\binom{5000}{100}}-\frac{\binom{25}{1}\binom{4975}{99}}{\binom{5000}{100}}=1-0.602724-0.309026=0.08825\]
Thus, we see that the probability of a shipment being rejected from Vendor 2 is approxiamtely $8.83\%$
\end{enumerate}
\item[Problem 2]\hfill \\
Given that the produced samples follow a normal distribution, we may easily compute the probabilities associated with the part being defective. This computation is an integral of the CDF for the normal distribution,
\[P(X\leq x)=\int_{-\infty}^{x}\frac{1}{\sigma\sqrt{2\pi}}e^{-(t-\mu)^2/(2\sigma^2)}dt\]
For this problem, the engineer has computed that,
\[\mu=10.48,\ \sigma=0.0142\]
Thus, we have two computations that we must compute. The first being the probability that the dimension is less than 10.45. This computation is computed using $R$ and its built in CDF integrator to obtain,
\[P(dim \leq 10.45)=0.017314\]
The second computation is slightly more complex, as we want the area to the right of the curve, thus,
\[P(dim > 10.55)=1-p(dim\leq 10.55)\]
Using $R$,
\[P(dim > 10.55)=1-0.9999996=4.1204\times10^{-7}\]
The probability of failure is then,
\[P(dim \leq 10.45)+P(dim > 10.55)=0.017314+4.1204\times10^{-7}=0.01731466\]
\item[Problem 3]\hfill \\
The distribution of spots on these circuit boards follows a Poisson distribution. To begin the computation of probabilities, we must compute,
\[\hat{\lambda}=\frac{97}{1000}\]
From this, we may say that the number of flaws, $X\sim Poisson(\hat{\lambda})$. Thus, we may use the PMF of the poisson distribution to compute,
\[P(X=x)=\frac{e^{-\hat{\lambda}}\hat{\lambda}^x}{x!}\]
\begin{enumerate}
\item So,
\[P(X=0)=\frac{e^{-\hat{\lambda}}\hat{\lambda}^0}{0!}=e^{-\hat{\lambda}}=0.907556\]
and
\[P(X=1)=\frac{e^{-\hat{\lambda}}\hat{\lambda}^1}{1!}=\hat{\lambda}e^{-\hat{\lambda}}=0.088033\]
Then,
\[P(pass)=P(X=0)+P(X=1)=0.99559\]
\item If the customer orders 500 Boards, we would expect that,
\[500*P(pass)=497.79\approx 498\]
Boards will pass inspection.
\end{enumerate}
\end{description}
\end{document}

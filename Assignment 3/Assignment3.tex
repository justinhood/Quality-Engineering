\documentclass[letterpaper,10pt]{article}
\usepackage[top=2cm, bottom=1.5cm, left=1cm, right=1cm]{geometry}
\usepackage{amsmath, amssymb, amsthm,graphicx,enumitem}
\usepackage{fancyhdr}
\pagestyle{fancy}

\lhead{\today}
\chead{Quality Engineering Assignment 3}
\rhead{Justin Hood}

\newcommand{\Z}{\mathbb{Z}}
\newcommand{\Q}{\mathbb{Q}}
\newcommand{\R}{\mathbb{R}}
\newcommand{\C}{\mathbb{C}}
\newtheorem{lem}{Lemma}

\begin{document}
\begin{description}
\item[Question 1.]\hfill \\
\begin{enumerate}[label=\alph*.]
\item Given that the process is modeled by a Poisson distribution with $\lambda=0.02$,
\[P(X=1)=\frac{(0.02)^1e^{-0.02}}{1!}\approx 0.019604\]
\item We now compute the probability of one or more defects as,
\[P(X\geq 1)=1-P(X=0)=1-\frac{(0.02)^0e^{-0.02}}{0!}=1-e^{-0.02}\approx 0.019801\]
\item Given that our new mean is $\lambda=0.01$, we again compute
\[P(X\geq 1)=1-P(X=0)=1-\frac{(0.01)^0e^{-0.01}}{0!}=1-e^{-0.012}\approx 0.00995017\]
This new probability is effectively half of our original probability.
\end{enumerate}
\item[Question 2.]\hfill \\
Consider the random variable with pdf:
\[p(X=x)=\begin{cases}
\frac{1+3k}{3} & x=1\\
\frac{1+2k}{3} & x=2\\
\frac{.5+5k}{3} & x=3
\end{cases}\]
We now note the following,
\begin{align*}
\sum_{x}P(X=x)&=1 && \Rightarrow\\
1 &= \frac{1+3k}{3}+\frac{1+2k}{3}+\frac{.5+5k}{3} && \Rightarrow\\
3 &= 10k+\frac{5}{2} && \Rightarrow\\
k &= \frac{1}{20}
\end{align*}
\item[Question 3.]\hfill \\
We note that this process can be modeled by a binomial distribution with $p=0.01$ and $n=25$. This process is stopped when $x\geq 1$. Thus,
\[P(x\geq 1)=1-P(X=0)=1-{25\choose 0}(0.01)^0(1-0.01)^{25-0}=1-0.77782\approx 0.2222\]
Hence, we see that this decision model will call for a stoppage around $22.2\%$ of the time. For a company to have a fault or stoppage over one fifth of the time would be quite expensive and time consuming.
\item[Question 4.]\hfill \\
The new distribution is now a binomial with $p=0.04$. Given that the process is stopped for one or more defects, we compute the probability of a stop as,
\[P(x\geq 1)=1-P(X=0)=1-{25\choose 0}(0.04)^0(1-0.04)^{25-0}\approx 0.639603\]
The average number of runs then is,
\[N=\frac{1}{0.639603}=1.56347\]
Rounding, we see that this defect should be found in around 2 runs of the process with the new defect rate.
\item[Question 5.]\hfill \\
Given that the strength is normally distributed, we may compute the probability that a part does not meet the minimum limit of 35 pounds of tensile strength with,
\[P(x<35)=\int_{-\infty}^{35}\frac{e^{-(x-40)^2/50}}{5\sqrt{2\pi}}\approx 0.158655\]
Given that we have sampled $50000$ parts, we expect,
\[E(x)=50000*0.158655=7932.76\to 7933\]
Parts to be defective.\\
Considering parts that have strength over 48 pounds, we compute,
\[P(x>48)=1-\int_{-\infty}^{48}\frac{e^{-(x-40)^2/50}}{5\sqrt{2\pi}}=1-0.945201\approx 0.0547993\]
Which results in an expected number of parts,
\[E(x)=50000*0.0547993=2739.96\to 2740\]
\item[Question 6.]\hfill \\
Given the normal distribution of the voltage, we may compute the probability of the voltage being within the stipulations by computing,
\[P(4.95\leq x \leq 5.05)=\int_{4.95}^{5.05}\frac{e^{-(x-5)^2/(2*.02^2)}}{.02\sqrt{2\pi}}\approx 0.987581\]
To find the variance required to reduce the failure rate, we consider that the probability of a failure from being lower than the minimum value is, $\frac{1}{2}\frac{1}{1000}$ Due to symmetry. So, we know that our z-score equation should be,
\begin{align*}
P(Z<\frac{4.95-5}{\sigma})&=0.0005\\
\frac{4.95-5}{\sigma}) &= -3.29\\
\sigma &= \frac{-.05}{-3.29}\\
&= 0.0152
\end{align*}
Hence, our variance would be,
\[\sigma^2= 2.3104\times 10^{-4}\]
\item[Question 7.]\hfill \\
To begin, we consider process 1. A normally distributed process with mean 7500 and deviation of 1000h as specified. We then compute the probability of a non usable part as,
\[P(reject)=1-\int_{5000}^{10000}\frac{e^{-(x-7500)^2/(2*1000^2)}}{1000\sqrt{2\pi}}\approx 0.01242\]
So, we consider the cost of using process 1.
\[C_1=Construction\ Cost+Repair\ Cost=P*n+5*.01242*n=(P+.0621)n\]
Similarly, for process 2, we compute the same values,
\[P(reject)=1-\int_{5000}^{10000}\frac{e^{-(x-7500)^2/(2*500^2)}}{500\sqrt{2\pi}}\approx 0\]
The Cost is then,
\[C_2=2P*n+5*0*n=2Pn\]
So, we may now compare the production costs to find the transition point in terms of production cost,
\begin{align*}
2Pn&=(P+0.0621)n\\
2P &= P+0.0621\\
P&=0.0621
\end{align*}
So, the manufacturer decision is then,
\[\begin{cases}
0<P<0.0621 & \text{Manufacturer 2}\\
P>0.0621 & \text{Manufacturer 1}
\end{cases}\]
\end{description}
\end{document}
